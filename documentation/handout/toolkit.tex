%http://www.maths.manchester.ac.uk/~kd/latextut/pdfbyex.htm
\documentclass[a4paper,twoside]{book}      % Comments after  % are ignored
\usepackage{amsmath,amssymb,amsfonts} % Typical maths resource packages
\usepackage{graphics}                 % Packages to allow inclusion of graphics
\usepackage{graphicx}
\usepackage{color}                    % For creating coloured text and background
\usepackage{hyperref}                 % For creating hyperlinks in cross references
\usepackage{listings}

\usepackage{color}
\usepackage{listings}
%\usepackage{textcomp}
\usepackage{setspace}
%\usepackage{palatino}

\newenvironment{codelisting0}
{\begin{list}{}{\setlength{\leftmargin}{1em}}\item\tiny}
{\end{list}}

\renewcommand{\lstlistlistingname}{Code Listings}
\renewcommand{\lstlistingname}{Code Listing}
\definecolor{orange}{rgb}{1,0.5,0}
\definecolor{gray}{gray}{0.5}
\definecolor{green}{rgb}{0,0.5,0}
\definecolor{lightgreen}{rgb}{0,0.7,0}
\definecolor{purple}{rgb}{0.5,0,0.5}
\definecolor{darkred}{rgb}{0.5,0,0}
\lstnewenvironment{code}[1][]{
\lstset{ %
language=XML,                % choose the language of the code
basicstyle=\footnotesize,       % the size of the fonts that are used for the code
numbers=left,                   % where to put the line-numbers
numberstyle=\footnotesize,      % the size of the fonts that are used for the line-numbers
stepnumber=1,                   % the step between two line-numbers. If it's 1 each line will be numbered
numbersep=5pt,                  % how far the line-numbers are from the code
backgroundcolor=\color{black},  % choose the background color. You must add \usepackage{color}
showspaces=false,               % show spaces adding particular underscores
showstringspaces=false,         % underline spaces within strings
showtabs=false,                 % show tabs within strings adding particular underscores
frame=single,                   % adds a frame around the code
tabsize=2,                      % sets default tabsize to 2 spaces
captionpos=b,                   % sets the caption-position to bottom
breaklines=true,                % sets automatic line breaking
breakatwhitespace=false,        % sets if automatic breaks should only happen at whitespace
caption = OWL encoding of a small persons and movie genres ontology.,
label=lst:owlExample
}}{}

\newenvironment{codelisting}
{\begin{list}{}{\setlength{\leftmargin}{1em}}\item}
{\end{list}}

\definecolor{darkgray}{rgb}{0.95,0.95,0.95}
\lstset{language=Java}
%\lstset{basicstyle=22}
\lstset{backgroundcolor=\color{darkgray}}
\lstset{numbers=left, numberstyle=\tiny, stepnumber=1, numbersep=5pt}
\lstset{keywordstyle=\color{red}}
\lstset{showspaces=false}
\lstset{commentstyle=\color{green}}
\lstset{stringstyle=\color{blue}}

\newenvironment{mytinylisting}
{\begin{list}{}{\setlength{\leftmargin}{1em}}\item\tiny\bfseries}
{\end{list}}




\oddsidemargin 0cm
\evensidemargin 0cm

\pagestyle{myheadings}         % Option to put page headers
                               % Needed \documentclass[a4paper,twoside]{article}
\markboth{{\small\it TUDIIR Toolkit Overview}}
{{\small\it David Urbansky, Klemens Muthmann} }

\textwidth 15.5cm
\textheight 24cm
%\textwidth 10.0cm
%\textheight 17cm
\topmargin -1cm
\parindent 0cm

\parskip 1mm
\newtheorem{theorem}{Theorem}[section]
\newtheorem{proposition}[theorem]{Proposition}
\newtheorem{corollary}[theorem]{Corollary}
\newtheorem{lemma}[theorem]{Lemma}
\newtheorem{remark}[theorem]{Remark}
\newtheorem{definition}[theorem]{Definition}

\def\R{\mathbb{ R}}
\def\S{\mathbb{ S}}

%\date{\small\it May 16, 2000}
\date{\today}
%\title{\fcolorbox{red}{blue}{\color{white}Including colour, pdf graphics and hyperlinks
%\footnote{A demonstration example including colored text and graphics}}}
\title{TUDIIR Toolkit Overview}
\author{David Urbansky, Klemens Muthmann \\
{\small TU Dresden, Department of Systems Engineering, Chair Computer Networks, IIR Group, Germany}
}
%\author{David Urbansky \\
%{\small TU Dresden, Department of Systems Engineering, Chair Computer Networks, IIR Group, Germany}
%}
%\author{Klemens Muthmann \\
%{\small TU Dresden, Department of Systems Engineering, Chair Computer Networks, IIR Group, Germany}
%}
\begin{document}
\maketitle
%\begin{abstract}
%This document explains basic functionalities of the TUDIIR Toolkit. Its intention is to explain (new) developers how to work with the toolkit and how to extend its capabilities.
%\end{abstract}

\tableofcontents

\chapter{Introduction}
Internet Information Retrieval (IIR) is a research domain in computer science that is concerned with the retrieval, extraction, classification, and presentation of information from the Internet. This toolkit provides functionality which is often needed to perform IIR tasks such as crawling, classification, and extraction of various information types.

\section{What is TOOLKIT?}
%TODO contributions
\section{What is TOOLKIT NOT?}
%TODO

\section{License}
The complete source code is licensed under the Apache License 2.0. All source files should include the following license snippet at the very top.

\begin{verbatim}
Copyright 2010 David Urbansky, Klemens Muthmann
Licensed under the Apache License, Version 2.0 (the "License"); you may not
use this file except in compliance with the License. You may obtain a copy of
the License at

http://www.apache.org/licenses/LICENSE-2.0

Unless required by applicable law or agreed to in writing, software
distributed under the License is distributed on an "AS IS" BASIS, WITHOUT
WARRANTIES OR CONDITIONS OF ANY KIND, either express or implied.
See the License for the specific language governing permissions and
limitations under the License.
\end{verbatim}

\section{Alternative and Complimentary Toolkits}
Some functionalities of this toolkit are covered in other libraries. Before you start using TOOLKIT you might want to take a look at these alternatives. The objective of TOOLKIT is to create new functionalities or improve existing ones, we do not intend to reinvent the wheel. For example, TOOLKIT has no functionality for part-of-speech tagging (PoS tagging) since there are many toolkits that do this task pretty accurate already. In case you do not find the functionality you are looking for in this toolkit, you probably will in one of the following toolkits.

\begin{enumerate}
\item \textbf{AlchemyAPI} \cite{alchemyapi} is a commercial web-service that can be used via several programming languages. The service offers named entity recognition, text classification, language identification, concept tagging, keyword extraction, content scraping, and web page cleaning.
The service comes in 4 variants: free, basic, professional, and metered.
\item \textbf{Apache Mahout} \cite{settings2apache} is a Java-based machine learning library. Its main features are collaborative filtering, user and item based recommenders, (fuzzy k-means clustering, mean shift clustering, latent dirichlet process allocation, singular value decomposition, parallel frequent pattern mining, complementary naive bayes classifier, and a random forest decision tree based classifier.
The library is licensed under the Apache Software license.
\item \textbf{Balie} \cite{balie} is a Java-based information extraction library. Its main features are language identification, tokenization, sentence boundary detection, and named entity recognition (using dictionaries).
The library is licensed under the GNU GPL and supports English, German, French, Romanian, and Spanish as input languages.
\item \textbf{ContentAnalyst} \cite{contentanalyst} is a commercial platform for text analytics. The platform's main features are concept search, dynamic clustering, near-duplicate document identification, automatic summarization, text classification, and latent semantic indexing.
\item The \textbf{Dragon Toolkit} \cite{zhou2007dragon} is a Java-based development package for information retrieval and text mining. Its main features are text classification, text clustering, text summarization, and topic modeling.
\item \textbf{FreeLing} \cite{atserias2006freeling} is a natural language processing library written in C++. Its main features are Text tokenization, sentence splitting, morphological analysis, sSuffix treatment, retokenization of clitic pronouns, flexible multiword recognition, contraction splitting, probabilistic prediction of unkown word categories, named entity detection, recognition of dates, numbers, ratios, currency, and physical magnitudes, PoS tagging, chart-based shallow parsing, named entity classification,  WordNet based sense annotation and disambiguation, Rule-based dependency parsing, and nominal correference resolution.
It is licensed under GPL and supports Spanish, Catalan, Galician, Italian, English, Welsh, Portuguese, and Asturian as languages. An online demo is available under \url{http://garraf.epsevg.upc.es/freeling/demo.php}.
\item \textbf{GATE} \cite{cunningham2002gate} is a Java-based text mining and processing framework. The framework itself comes with few text processing features but many plugins can be used and chained into a text engineering pipeline.
The framework is licensed under the GNU Lesser General Public License.
\item The \textbf{Illinois Cognitive Computation Group} \cite{illinoisccg} has a list of ready to use programs for semantic role labeling, text chunking, named entity tagging, named entity discovery, PoS tagging, unsupervised rank aggregation, and named entity similarity metrics.
\item \textbf{Julie NLP} \cite{tomanek2007uima} is a Java-based toolkit of UIMA based text processing components. The toolkit can be used for semantic search, information extraction, and text mining.
The Toolkit is licensed under the Common Public License.
\item \textbf{Language Computer} \cite{lanuagecomputer} provides commercial products for sentence splitting, tokenization, PoS tagging, named entity recognition, co-reference resolution, attribute extraction, relationship extraction, event extraction, question answering, and text summarization.
\item \textbf{Lingo3G} \cite{lingo3g} is a text clustering engine that organizes text collections into hierarchical clusters.
The software is commercial but \cite{stefanowski2003carrot} offers an open source alternative for text clustering algorithms written in Java. The algorithms integrate with other programming or scripting languages such as PHP, Ruby, and C\verb$#$ too.
\item \textbf{LingPipe} \cite{lingpipe} is a text processing toolkit using computational linguistics. LingPipe is written in Java. Its main features are topic classification, named entity recognition, clustering, PoS tagging, sentence detection, spelling correction, database text mining, string comparisons, interesting phrase detection, character language modeling, chinese word segmentation, hyphenation and syllabification, sentiment analysis, language identification, singular value decomposition, logistic regression, expectation maximization, and word sense disambiguation.
LingPipe is available under a free license for academic use and several commercial licenses.
\item \textbf{Mallet} \cite{mccallum2002mallet} is a Java-based toolkit for statistical natural language processing. Its main features are text classification, sequence tagging (PoS tagging), topic modeling, and numerical optimization.
The toolkit is licensed under the Common Public License.
\item \textbf{MinorThird} \cite{cohen2004minorthird} is a Java-based toolkit for text processing. Its main features are annotating text, named entity recognition, and text classification.
The toolkit is licensed under the BSD license.
\item \textbf{MontyLingua} \cite{liu2004montylingua} is a Python and Java-based toolkit for natural language processing (English only). Its main features are tokenization, PoS tagging, lemmatization, and natural language summarization.
The toolkit is free for non-commercial use and licensed under the MontyLingua version 2.0 License.
\item \textbf{MorphAdorner} \cite{morphadorner} is a Java-based command line program for text processing. Its main features are language recognition, lemmatization, name recognition, PoS tagging, noun pluralization, sentence splitting, spelling standardization, text segmentation, verb conjugation, and word tokenization.
The program is licensed under a NCSA style license.
\item \textbf{NaCTeM Software Tools} \cite{nactem} are programs for natural language processing and text mining that are made available by the National Centre for Text Mining. The programs include functionality for PoS tagging, syntactic parsing, named entity recognition, sentence splitting, text classification, and sentiment analysis.
\item \textbf{NLTK} \cite{loper2002nltk} is a Python-based natural language processing toolkit. Its main features are tokenization, stemming, PoS tagging, text classification, and syntactic parsing.
The toolkit is licensed under the Apache 2.0 license.
\item \textbf{OpenCalais} \cite{opencalais} is a web service that performs named entity recognition, fact and event extraction.
The web service is free for commercial and non-commercial use but limited to 50,000 transactions a day. A professional plan is available too including more transactions and an service license agreement.
\item The \textbf{RASP System} \cite{briscoe2006second} is a C and Lisp-based toolkit for natural language processing (English only). Its main features are tokenization, PoS tagging, lemmatization, morphological analysis, and grammar-based parsing.
The toolkit is free for non-commercial use and licensed under the RASP System License.
\item The \textbf{Rosette Linguistic Platform} \cite{rosette} is a software suite that can perform name translation, name matching, named entity recognition, morphological analysis, and language identification. The suite works for 55 European, Asian, and Arabic languages.
The software is a commercial product.
\item \textbf{Stanford NLP} \cite{stanfordnlp} is a set of Java-based natural language processing libraries. Their main features are PoS tagging, named entity recognition, Chinese word segmentation, and classification.
The software distributions are licensed under the GNU Public License.
\item \textbf{SRILM - The SRI Language Modeling Toolkit} \cite{stolcke2002srilm} is a C++-based toolkit for language modeling. Its main features are speech recognition, statistical tagging and segmentation, and machine translation.
The toolkit is free for non-commercial use and licensed under the SRILM Research Community License.
\item \textbf{TextAnalyst} \cite{textanalyst} is a commercial text processing software offering text summarization, semantic information retrieval, meaning extraction, and text clustering.
\item \textbf{VisualText} \cite{visualtext} is a natural language processing software that addresses named entity recognition, text indexing, text filtering, text classification, text grading, and text summarization.
\item \textbf{WEKA} \cite{hall2009weka} is a Java-based machine learning and data mining library. The library contains a large set of machine learning algorithms such as Support Vector Machines, Neural Networks, Naive Bayes, k-nearest neighbor for but not limited to (text) clustering, (text) classification, and regression.
The library is licensed under the GNU General Public License.
\end{enumerate}

\chapter{Installation and making it work}
The TUDIIR toolkit is managed using \href{http://subversion.apache.org/}{Subversion (SVN)} (see~\ref{sec:toolkitstructure}). It is build and tested automatically on a weekly basis using \href{http://maven.apache.org/}{Apache Maven} and \href{http://hudson-ci.org/}{Hudson CI}. Bugs might be reported using the \href{http://www.mantisbt.org/}{Mantis bugtracker}. The project encoding must be set to UTF-8. To support unavailable libraries we manage our own Maven repository using \href{http://nexus.sonatype.org/}{Sonatype Nexus}. How to use these components is explained in the next sections.

At the moment each of the systems manages its own user base so you need to register with each one individually. When you start working with the toolkit you might consider sending an E-Mail to the administrator to get access to Mantis, Nexus and Hudson. Provide your name, the name of your advisor and the reason why you need to work with the toolkit and you will get your logins.
\section{Building the toolkit using Apache Maven}
\label{sec:buildingthetoolkitusingapachemaven}
Apache \textbf{Maven needs to be installed} before building the toolkit. How to do this is explained here: \href{http://maven.apache.org/download.html#Installation}{Maven installation}. There is one necessary manual step before you can start building the toolkit. You need to add your personal credentials for the TU Nexus repository to your local maven settings. Do this by \textbf{locating or creating the settings.xml file in your local home folder:} \texttt{\%YOUR\_HOME\_FOLDER\%/.m2/settings.xml} and adding the following content:
\begin{verbatim}
<settings xmlns="http://maven.apache.org/SETTINGS/1.0.0"
  xmlns:xsi="http://www.w3.org/2001/XMLSchema-instance"
  xsi:schemaLocation="http://maven.apache.org/SETTINGS/1.0.0
                      http://maven.apache.org/xsd/settings-1.0.0.xsd">
  <servers>
    <server>
      <id>nexus</id>
      <username>your-username</username>
      <password>your-password</password>
      <filePermissions>664</filePermissions>
      <directoryPermissions>775</directoryPermissions>
      <configuration></configuration>
    </server>
  </servers>
</settings>
\end{verbatim}
You need to set your username and your password. This can be obtained by sending an E-Mail to \href{mailto:klemens.muthmann@tu-dresden.de}{klemens.muthmann@tu-dresden.de} stating who is your advisor and why you need to work with the toolkit. After completing the installation, perform the following steps to build the toolkit using Maven:
\begin{enumerate}
\item Check out the code from SVN!
\item Open your favorite command line.
\item Change to the toolkits root folder.
\item Type \texttt{mvn clean install} to start the build process.
\end{enumerate}
There also is an \href{http://m2eclipse.sonatype.org/}{Eclipse plugin}, that allows you to issue maven build from within Eclipse. It is quite beta so be careful with it.
\section{Regularly builds and tests using Hudson CI}
Currently the toolkit is build automatically every week using Hudson CI. Be careful to check in only working code or Hudson will send you and your advisor an E-Mail about broken code. If this happens try to fix your code as fast as possible and check in again. You can get details about the problem by logging into \href{http://www.effingo.de/hudson}{Hudson}. You can get a login by sending an E-Mail stating your advisor and why you nood to work with the toolkit to \href{mailto:klemens.muthmann@tu-dresden.de}{klemens.muthmann@tu-dresden.de}
\subsection{The Continuous Integration Game}
\label{sec:cigame}
Hudson supports a game that rewards people commiting code that improves the toolkit and punishing people breaking it. The leader (the person having the most points) is highly valued by the toolkit commiters community. The rules for the game are explained in detail in the next section.
\paragraph{Rules}
The rules of the game are:
\begin{itemize}
\item -10 points for breaking a build
\item 0 points for breaking a build that already was broken
\item +1 points for doing a build with no failures (unstable builds gives no points)
\item -1 points for each new test failures
\item +1 points for each new test that passes
\item Adding/removing a HIGH priority PMD warning = -5/+5. Adding/removing a MEDIUM priority PMD warning = -3/+3. Adding/removing a LOW priority PMD warning = -1/+1.
\item Adding/removing a violation = -1/+1. Adding/removing a duplication violation = +5/-5.
\item Adding/removing a HIGH priority findbugs warning = -5/+5. Adding/removing a MEDIUM priority findbugs warning = -3/+3. Adding/removing a LOW priority findbugs warning = -1/+1
\item Adding/removing a compiler warning = -1/+1.
\item Checkstyle Plugin. Adding/removing a checkstyle warning = -1/+1.
\end{itemize}

\section{Hello Toolkit - Your first application using the TUDIIR Toolkit}
\paragraph{Project creation:} Create a new Maven Project using the New Project wizard of Eclipse (See Fig.~\ref{fig:maven-project01},~\ref{fig:maven-project02} and \ref{fig:maven-project03}).
\begin{figure}
\includegraphics[width=\textwidth]{img/ht01.png}
\caption{Create a new Maven Project.}
\label{fig:maven-project01}
\end{figure}
\begin{figure}
\includegraphics[width=\textwidth]{img/ht02.png}
\caption{Choose to create a simple Maven Project.}
\label{fig:maven-project02}
\end{figure}
\begin{figure}
\includegraphics[width=\textwidth]{img/ht03.png}
\caption{Enter detail information about your new Maven Project}
\label{fig:maven-project03}
\end{figure}
\paragraph{Adding the toolkit dependency to the project:} Right click on the new project. In the context menu that appears choose \textit{Maven $\rightarrow$ Add Dependency} (See Fig.~\ref{fig:example-project-context-menu01} and \ref{fig:example-project-context-menu02}).
\begin{figure}
\includegraphics[scale=1]{img/context02.png}
\caption{Maven Project Context Menu. Choose \textit{Maven}}
\label{fig:example-project-context-menu01}
\end{figure}
\begin{figure}
\includegraphics[scale=1]{img/context01.png}
\caption{Create a new Maven project}
\label{fig:example-project-context-menu02}
\end{figure}
A search interface appears (See~\ref{fig:add-dependency01}). If you followed the steps in Section~\ref{sec:buildingthetoolkitusingapachemaven} and installed the toolkit to your local Maven repository, you can type \textit{toolkit} in the search interface (See Fig.~\ref{fig:add-dependency02} and add the dependency with a double click on the \textit{de.tud.inf.rn.iir toolkit} entry. 

Note: If you need to add further dependencies in the future you can use the same steps.
\begin{figure}
\includegraphics[width=\textwidth]{img/ht04.png}
\caption{Search interface for Maven dependencies.}
\label{fig:add-dependency01}
\end{figure}
\begin{figure}
\includegraphics[width=\textwidth]{img/ht05.png}
\caption{Search results for \textit{toolkit}}
\label{fig:add-dependency02}
\end{figure}
The Maven plugin adds the dependency to your pom.xml file, which should look like Fig.~\ref{fig:pom01}.
\begin{figure}
\includegraphics[width=\textwidth]{img/ht06.png}
\caption{POM with added dependency on the TUD IIR Toolkit.}
\label{fig:pom01}
\end{figure}
\paragraph{Configuring your project} Since Maven by default still uses Java 1.4 (very conservative), but the toolkit depends on Java 1.6 (very visionary) you need to configure the Maven Java compiler plugin to use Java 1.6. This is shown in Fig.~\ref{fig:java601}.
\begin{figure}
\includegraphics[width=\textwidth]{img/ht07.png}
\caption{Configure the Maven Java compiler to use Java 1.6 instead of 1.4}
\label{fig:java601}
\end{figure}
The same step is necessary to tell Eclipse that it should use Java 1.6 instead of 1.4. For this purpose open the project properties via the context menu for example. In the tree to the left choose \textit{Java Compiler} and change all three entries to 1.6. This is shown in Fig.~\ref{fig:java602} and Fig.~\ref{fig:java603}.
\begin{figure}
\includegraphics[width=\textwidth]{img/ht08.png}
\caption{Eclipse uses Java 1.4 by default for Maven projects}
\label{fig:java602}
\end{figure}
\begin{figure}
\includegraphics[width=\textwidth]{img/ht09.png}
\caption{configure Eclipse to use Java 1.6}
\label{fig:java603}
\end{figure}
\paragraph{Writing your first toolkit code} Now you can start to write your first code. Your project will already contain the default Maven directory structure. Do not change this structure since Maven depends on it\footnote{Of course you can change Mavens behaviour via the pom.xml but this requires additional configuration not covered by this document. Refer to the Maven documentation under \url{http://maven.apache.org/} for further information.}. As usual we will start with a very simple "Hello World" application. Create a new package \texttt{de.tud.inf.rn.iir.toolkit} and a new class \texttt{HelloToolkit} containing a \texttt{main} method. In this main method you can add actual toolkit code. We used the first example as described in Section~\ref{sec:howto} and search Bing for the term "Hello Toolkit". The example is also shown in Fig.~\ref{fig:hellotoolkit}.
\begin{figure}
\includegraphics[width=\textwidth]{img/ht10.png}
\caption{"Hello Toolkit" Code}
\label{fig:hellotoolkit}
\end{figure}
For the code to work you need three additional files that are already in the config folder in the toolkit project. These are ``crawler.conf'', ``apikeys.conf'' and ``feeds.conf''. You need to copy them to src/main/resources/config. The folder src/main/resources usually contains all resources that are not Java files but are required by your code. All files are shown in Fig.~\ref{fig:resource01}, \ref{fig:resource02} and \ref{fig:resource03}. Fig.~\ref{fig:structure} shows the final directory and file structure of your project.
\begin{figure}
\includegraphics[width=\textwidth]{img/ht11.png}
\caption{Create a new Maven project}
\label{fig:resource01}
\end{figure}
\begin{figure}
\includegraphics[width=\textwidth]{img/ht12.png}
\caption{Create a new Maven project}
\label{fig:resource02}
\end{figure}
\begin{figure}
\includegraphics[width=\textwidth]{img/ht13.png}
\caption{Create a new Maven project}
\label{fig:resource03}
\end{figure}
\begin{figure}
\includegraphics[width=\textwidth]{img/ht14.png}
\caption{Create a new Maven project}
\label{fig:structure}
\end{figure}
With the structure from Fig.~\ref{fig:structure} you can run your project. Just open your projects context menu again, choose \textit{Run As $\rightarrow$ Maven Clean}, then open context menu again choose \textit{Run As $\rightarrow$ Maven Install} and finally choose \textit{Run As $\rightarrow$ Java Application}.

\section{Reporting issues using Mantis}
To report issues with the toolkit or to view issues assigned to you, you need to register with Mantis. To do this send an E-Mail to the administrator at \href{mailto:klemens.muthmann@tu-dresden.de}{klemens.muthmann@tu-dresden.de} and provide your name, your advisor and the reason why you are working on the toolkit (i.e. thesis topic or research fellow). You will receive an E-Mail (not automatically so it might take some time) providing a link were you can set your password. If you successfully set a password for your account you can login on: \href{http://141.76.40.242/mantisbt}{Mantis}.
%\subsection{Building the toolkit using Apache Ant}
%You can also build the toolkit using Apache Ant\footnote{\url{http://ant.apache.org/}}. To do so, install Apache Ant, put the binary folder to your PATH and compile the project from the root folder that contains the build.xml file.
%The following tasks can be performed by ant (type commands into your console):
%\begin{itemize}
%\item \textbf{standard} Type ``ant'' to clean and compile the project.
%\item \textbf{jar} Type ``ant jar'' to pack TUDIIR into one single jar file. That file can then be found under target/jar/tudiir.jar.
%\item \textbf{javadoc} Type ``ant javadoc'' to (re)create the javadoc which can then be found under documentation/javadoc. You  need to install Graphviz\footnote{\url{http://www.graphviz.org/}} beforehand. See Java Dzone\footnote{\url{http://java.dzone.com/articles/reverse-engineer-source-code-u}} for more information.
%\end{itemize}
%Apache Ant will create a target folder, please do not commit that folder to the repository but ignore it using svn:ignore.

\chapter{Toolkit Structure}
\label{sec:toolkitstructure}
The TUDIIR Toolkit is managed using subversion. The top level folder structure follows the usual subversion layout using trunk for the main development, branches for parallel development and tags to mark specific versions. The trunk is located in our SVN\footnote{\url{https://141.76.40.86/svn-students/iircommon/toolkit/trunk}}. The folder structure looks as follows.
\begin{verbatim}
toolkit
 |- config
 |- data
    |- knowledgeBase
    |- models
    |- test
 |- documentation
    |- handout
    |- javadoc
    |- other
 |- exe
 |- libs
 |- src
    |- main
       |- java
    |- test
       |- java
 |- dev
\end{verbatim}

\section{Config Folder}
\label{sec:config.conf}
The config folder contains configuration files for several components of the toolkit. The files are explained in the following sections.

\subsection{apikeys.conf}
\label{sec:apikeys.conf}
The api keys that are used by the toolkit components are specified here. You may need to apply for API keys at the provider's page.

\begin{verbatim}
yahoo.api.key = 
yahoo_boss.api.key = 
hakia.api.key = 
google.api.key = 
bing.api.key = 
\end{verbatim}

\subsection{classification.conf}
\label{sec:classification.conf}
Classification settings for the tud.iir.classification.page.ClassifierManager.

\begin{verbatim}
# percentage of the training/testing file to use as training data
page.trainingPercentage = 80

# create dictionary on the fly (lowers memory consumption but is slower)
page.createDictionaryIteratively = false

# alternative algorithm for n-gram finding (lowers memory consumption but is slower)
page.createDictionaryNGramSearchMode = true

# index type of the classifier:
# 1: use database with single table (fast but not normalized and more disk space needed)
# 2: use database with 3 tables (normalized and less disk space needed but slightly slower)
# 3: use lucene index on disk (slow)
page.dictionaryClassifierIndexType = 1
\end{verbatim}

\subsection{crawler.conf}
\label{sec:crawler.conf}
Crawler settings for the tud.iir.web.Crawler. All settings can be set in Java code as well.

\begin{verbatim}
# maximum number of threads during crawling
maxThreads = 10

# stop after x pages have been crawled, default is -1 and means unlimited
stopCount = -1

# whether to crawl within a certain domain, default is true
inDomain = true
	
# whether to crawl outside of current domain, default is true
outDomain = true

#  number of request before switching to another proxy, default is -1 and means never switch
switchProxyRequests = -1
	
# list of proxies to choose from
proxyList = 83.244.106.73:8080
proxyList = 83.244.106.73:80
proxyList = 67.159.31.22:8080
\end{verbatim}

\subsection{db.conf}
Database settings for the tud.iir.persistence.DatabaseManager.

\begin{verbatim}
db.type = mysql
db.driver = com.mysql.jdbc.Driver
db.host = localhost
db.port = 3306
db.name = toolkitdb
db.username = root
db.password = rootpass
\end{verbatim}

\subsection{general.conf}
General settings used by the tud.iir.control.Controller.

\section{Data Folder}
The data folder contains files that are used during runtime of several components.

\subsection{knowledgeBase} 
The knowledge base folder contains the OWL ontology files used for the extraction tasks in the tud.iir.extraction package.

\subsection{models}
The models folder contains learned models that can be reused.

\subsection{Temp Folder}
The temp folder is not part of the repository. Some functions may however create this folder and write temporary data.

\subsection{test}  
The test folder contains data that is used for running jUnit tests.

\section{Documentation Folder}
The documentation folder contains help files to understand the toolkit. This very document is located in the handout folder and a Javadoc can be found there too.

\section{Exe Folder}
The exe folder contains all runnable jar files in separate folders including a sample script to run the program and a readme.txt that explains the run options.

\section{Libs Folder}
The libs folder contains all referenced libs used by the toolkit.

\section{Src Folder}
The src folder contains all source files of the toolkit. You may need to put the log4j.properties file here in order to use custom logging settings. Alternatively you include the config folder in the class path.


\chapter{Conventions}
\section{Coding Standards}
To keep the code readable and easy to understand for other developers, we use the following coding guidelines.
%http://geosoft.no/development/javastyle.html
\begin{enumerate}
\item All text is written in (American) English (color instead of colour).
\item Variables and method names should be camel cased and not abbreviated ($computeAverage$ instead of $comp\_avg$).
\item There must be a space after a comma.
\item There must be a line break after opening $\{$ braces.
\item Static fields must be all uppercase. There should be an $\_$ for longer names ($STATIC\_FIELD$).
\item Each class must have a comment including the author name.
\item Methods with very simple, short code do not need to have comments (getters and setters) all other methods should have an explaining comment with @param explanation and @return.
\item Avoid assignments ($=$) inside if and while conditions.
\item Statements after conditions should always be in braces ($\{\}$)
\end{enumerate}

The following listing shows an example class with applied coding standards. Please also have a look at \cite{codingStandards} for a quick overview of best practices.

\begin{codelisting}
\begin{lstlisting}[frame=tb]
/**
 * This is just an example class.
 * It is here to show the coding guidelines.
 * 
 * @author Forename Name
 */
public class ExampleClass implements Example {

	// this field holds all kinds of brackets
	private static final char[] BRACKET_LIST = {'(', ')'};

	/**
	 * This is just and example method.
	 * 
	 * @param timeString A string with a time.
	 * @return True if no error occurred, false otherwise.
	 */
	public boolean computeAverageTime(String timeString) {
		if (hours < 24 && minutes < 60 && seconds < 60) {
			return true;
		} else {
			return false;
		}
	}
}
\end{lstlisting}
\end{codelisting}

\section{Eclipse Plugins for better Coding}
\label{sec:eclipseCodingPlugins}
It is a very good practice to install the following eclipse plugins to check ones own code before committing:
\begin{enumerate}
\item CodeFormatter is a simple file that can be loaded into Eclipse to format the source code correctly. Go to Eclipse $\blacktriangleright$ Window $\blacktriangleright$ Preferences $\blacktriangleright$ Java $\blacktriangleright$ Code Style $\blacktriangleright$ Formatter and import the ``tudiir\_eclipse\_formatter.xml'' from the dev folder. Pressing Control+F formats the source code of a selected class.
\item Checkstyle \footnote{\url{http://eclipse-cs.sourceforge.net/downloads.html}} checks styles of the code, whether JavaDoc comments are set etc. After installing you should go to the preferences section of Checkstyle in Enclipse and load the ``checkstyle\_config.xml'' from the dev folder.
\item PMD\footnote{\url{http://pmd.sourceforge.net/eclipse/}} tells you what is wrong with your code in terms of common violations, forgotten initializations and much more. After installing you should go into the preferences of PMD and load the ruleset file ``pmd\_ruleset'' from the dev folder.
\item FindBugs\footnote{\url{http://findbugs.cs.umd.edu/eclipse/}} is similar to PMD but focuses on severe errors only. After installing you don't need to configure anything.
\end{enumerate}

Checkstyle, PMD, and FindBugs can be initiated by right clicking a package or class file and selecting the plugin. The violations will be shown so that you can eliminate them.

Using these plugins raises the chances to win in the continuous integration game as described in Section \ref{sec:cigame}.

\subsection{Tests}
To guarantee that all components work as expected, we use jUnit tests. Before major check-ins to the repository, all jUnit Tests must run successfully. Run the tud.iir.control.AllTests.java to make sure all components work correctly.
After finishing a new component, new testing code must be written.

%\section{Project Management}
\chapter{Toolkit Functionality}
\section{Classification}

\subsection{Text Classification}
Text classification is the process of assigning one or more categories to a given text document. There are several types of text classification which are shown in Figure \ref{fig:typesOfClassification}. The TOOLKIT can classify text in a single category, multiple categories, or in a hierarchical manner.

\begin{figure}
\includegraphics[width=\textwidth]{img/typesOfClassification.png}
\caption{Types of classification\cite{qi2009web}.}
\label{fig:typesOfClassification}
\end{figure}

The text classification components are built from scratch and do not rely on external libraries such as Weka. In this section we will explain which features can be used for the classification, the basic theory of the classifiers, and how the performance of a classifier can be evaluated. In each section, we will describe theory and how the classification components can be used programmatically.

\subsubsection{Classification Type}
The TOOLKIT supports simple single category classification, tagging, and hierarchical classification. Each classifier needs to know in which type it has to perform classification.
This information is stored in the $tud.iir.classification.page.evaluation.ClassificationTypeSetting$ object which is then passed to the classifier. Please read the Javadoc of that class for more detailed information.

\subsubsection{Features}
Features are the input for a classifier. In text classification we have a long string as an input from which we can derive several features. All of the TOOLKIT classifiers work with n-grams. N-grams are sets of tokens of the length n. The TOOLKIT can preprocess text with character or word-level n-grams:

\begin{enumerate}
\item \textbf{Character level n-grams} use each character of the string as a token. For example, from the string ``It is sunny'' we can create the following set of 3-grams: ${it ,t i, is,is ,s s, su,sun,unn,nny}$. The number of n-grams in a set can be calculated as $ngrams = numberOfTokens - n + 1$. In our example that means $9 = 11 - 3 + 1$.
\item \textbf{Word level n-grams} use each word (separated with white space) of the string as a token. For example, from the string ``It is so nice and sunny today'' we can create the following set of 3-grams: ${It is so,is so nice,so nice and,nice and sunny,and sunny today}$. The number of n-grams in a set can be calculated as $ngrams = numberOfTokens - n + 1$. In our example that means $5 = 7 - 3 + 1$. If you want to have single words as features for the text classifier you can simply use unigrams or bigrams which are n-grams with $n=1$ or $n=2$ respectively.
\end{enumerate}

The document preprocessor allows you to create a set of n-grams with different length too. For example, you can create all 2-grams, 3-grams, and 4-grams for the given input text and use them as features for the classifier.

Sometimes you do not want to have all n-grams to be features for the classifier. You can simply disallow certain features by putting them in a stop word list. These n-grams will then be ignored for the classification.

%CODE
All these settings are stored in a $tud.iir.classification.page.evaluation.FeatureSetting$ object which is passed to the classifier. Please read the Javadoc of that class for more detailed information.

\subsubsection{Text Classifiers}
The text classifiers perform the actual classification task by calculating the most relevant category (or categories) for the input document. Two text classifiers are implemented, a dictionary based classifier and a k-nearest neighbor classifier. Both are explained in more detail in the following paragraphs.

\paragraph{K-Nearest Neighbor Text Classifier}
The KNN classifier uses the n-grams of the training documents to place them in a high dimensional vector space. The dimensions of the space equal the total number of available n-grams. Each training document is therefore a vector in that highly dimensional space. A new, unclassified document is now put into that vector space and by using distance function the k nearest neighbors are found for that document. Each of these neighbors votes with its own class, the more votes for one class the more likely that the new document belongs to that class.

\begin{figure}
\includegraphics[width=\textwidth]{img/knn.png}
\caption{A simple KNN example.}
\label{fig:knn}
\end{figure}

Figure \ref{fig:knn} shows a simple example of how the KNN classifier works. We limited the dimension to two for easier understanding. In this scenario we want to classify the given document (black dot in the middle) into one of the two categories ``finance'' (blue triangles) or ``travel'' (red dots). We calculate the distance between the new document and all training documents and consider the votes of the nearest three. In the example, two of these three document vote for ``travel'' which would let us classify our input document into that class.

The distance between two documents is calculated as shown in Equation \ref{eqn:knnDistance} where $d1$ and $d2$ are the two documents. The shorter the distance, the more similar the documents.

\begin{equation}
\label{eqn:knnDistance}
distance(d1,d2) = \frac{1}{numberOfMatchingNGrams}
\end{equation}

\paragraph{Dictionary-Based Classifier}
The dictionary-based classifier\footnote{This classifier won the first Research Garden (\url{http://www.research-garden.de}) competition where the goal was to classify product descriptions into 8 different categories.} learns how probable each n-gram is for each given category and assigns the most probable category (or categories) to the input document.

A dictionary is built at training stage by counting and normalizing the co-occurrences of one n-gram and a category. The dictionary might then look as shown in Table \ref{tab:dictionary} where each column is a category (finance, travel, and science) and each row is an n-gram. In each cell we now have the learned relevance for each n-gram and category $relevance(ngram,category)$. The sum of the relevances in each row must add up to one.

Table \ref{tab:dictionary} shows an example dictionary matrix. The n-gram ``money'' is more likely to get the category ``finance'' ($relevance(money,finance) = 0.6$) than ``science'' ($relevance(money,science) = 0.25$) while the n-gram ``beach'' is most likely to appear in the category ``travel'' ($relevance(beach,travel) = 0.85$.

\begin{table}[ht]
\centering
\caption{N-Gram dictionary with relevances for categories.}
\begin{tabular}{|l|l|l|l|}
\hline
n-gram   & finance & travel & science \\
\hline
money	   & 0.6	&	0.15	&	0.25	\\
\hline
beach	& 0.1	&	0.85	&	0.05	\\
\hline
paper	   & 0.3 &	0.2	&	0.5	\\
\hline
\end{tabular} 
\label{tab:dictionary}
\end{table}

To classify a new document, we again create all n-grams, look up the relevance scores in the dictionary and assign the categories with the highest probability. The probability for each category and given document is calculated as shown in Equation \ref{eqn:dictionaryCategoryProbabilty} where $N_{Document}$ is the set of n-grams for the given document.

\begin{equation}
\label{eqn:dictionaryCategoryProbabilty}
\mbox{$CategoryProbability(category,document)$} = \sum_{n\,\epsilon\, Nurl} \mbox{$relevance(n,category)$}
\end{equation}

The dictionary can be stored in memory, in an embedded H2 database, or in a client/server MySQL database. These settings can be made in the classification.conf file in the conf folder (see Section \ref{sec:config.conf}).

%TODO category boost
%TODO cooccurrence

\subsubsection{Evaluation}
In order to find out which classifier works best with which feature settings, you can evaluate these combinations. The $tud.iir.classification.page.ClassifierManager$ has the $learnBestClassifier$ method to run the evaluation on all given classifiers with the given evaluation setting object $tud.iir.classification.page.EvaluationSetting$. See Section \ref{sec:bpTextClassification} for more details of how to perform the evaluation programmatically.

The output of the evaluation will be three csv files that hold information about the combinations of classifier, dataset, training percentage, and the final performance for the combination. The performance is measured in precision, recall, and F1. The three files are stored in the data/temp folder and hold the following information.

\paragraph{averagePerformancesDatasetTrainingFolds.csv} This file holds the performance measures for each classifier, averaged over all given datasets, training percentages, and folds in the cross validation.

\paragraph{averagePerformancesTrainingFolds.csv} This file holds the performance measures for each classifier and dataset combination, averaged over all given training percentages and folds in the cross validation.

\paragraph{averagePerformancesFolds.csv} This file holds the performance measures for each classifier, dataset, and training percentage combination, averaged over all given folds in the cross validation.

\subsubsection{Best Practices}
\label{sec:bpTextClassification}

%TODO learn a classifier (and save it)
%TODO test a classifier
%TODO use a saved classifier
%TODO learn bestclassifier

\section{Extraction}
\subsection{Web Page Content Extraction}
Content oriented web pages such as blogs or news articles contain not only the text of interest but also clutter such as the navigation, footer, header, and ads. In order to automatically process the text without the clutter, we need to extract the text content. Figure \ref{fig:webpagecontentextractor} shows an example web page where the main article is in the green box. Everything else is just clutter and should not be extracted when looking for the unique article.

\begin{figure}[ht!]
\includegraphics[width=\textwidth]{img/webpagecontentextractor.png}
\caption{Create a new Maven Project.}
\label{fig:webpagecontentextractor}
\end{figure}

To perform this kind of extraction we could use wrapper approaches which are explained in more detail in \cite{ckcs06}. However, these approaches require manual wrapper construction or semi-supervised learning which is too cumbersome.
Another approach would be to detect the template of the web page and get only the contents of the main block (the green box). Template detection usually works by comparing one page with several other pages of the same domain to find the fix and changeable contents. This however requires several http requests and therefore more time and bandwidth.
In TOOLKIT we use an approach that is based on a hierarchical analyzis of the web page's DOM tree. Each DOM element is analyzed and ranked based on the length of the contained text, the link density, and the frequency of other elements in relation to the text length. Longer text fragments usually indicate a relevant part of the article, while a high link density is more likely to indicate that the analyzed element is part of the navigation for example. Also the attributes ``id'' and ``class'' are analyzed. If the attribute values contain keywords such as ``entry'', ``content'', or ``text'', the element is more likely to contain relevant content as if the attribute values contain keywords such as ``header'', ``footer'', or ``sidebar''. The implementation of the web page content extraction adopted great parts of the Firefox extension ``Readability''\footnote{\url{http://lab.arc90.com/experiments/readability/}}. The used heuristics in Readability have shown to be quite accurate for a wide range of web pages.

\subsection{Fact Extraction}
\subsection{Date Extraction}
\subsection{Named Entity Recognition}

\section{Retrieval}
\paragraph{Web Crawling}
\paragraph{Web Information Retrieval}
\paragraph{Feed Retrieval}

\section{Preprocessing}
\subsection{Sentence Splitting}
TOOLKIT has a rudimentary implementation for the common need for sentence splitting. TOOLKIT's implementation works with hand-crafted rules and thus does not require a model. Sentences try to be splitted on periods, question marks, and exclamation marks but there are also rules that try to prevent splitting sentences at ellipses. For example, the following is online one sentence although it contains several periods: ``Sometimes sentenes contain many periods...really!''.

The following code shows how the sentence splitting can be used:
\begin{codelisting}
\begin{lstlisting}[frame=tb]
String inputText = "This is a sentence. This is another one!";
List<String> sentences = Tokenizer.getSentences(inputText);
CollectionHelper.print(sentences);
// prints:
// This is a sentence
// This is another one!
\end{lstlisting}
\end{codelisting}

You can also get a specific sentence by providing a phrase that is part of the sentence using the getSentence method.

\subsection{Tokenization}
\subsection{Creating N-Grams}
\subsection{Noun Pluralization and Singularization}
TOOLKIT is able to transform most English singular nouns to their plural and back. For example, ``city'' becomes ``cities'' and ``index'' becomes ``indices''.

The following code shows the simple usage of the singularization and pluralization using the WordTransformer class.
\begin{codelisting}
\begin{lstlisting}[frame=tb]
String singular = "city";
String plural = "";
plural = WordTransformer.wordToPlural(singular);
singular = WordTransformer.wordToSingular(plural);
System.out.println(singular);
System.out.println(plural);
// prints:
// cities
// city
\end{lstlisting}
\end{codelisting}

\chapter{More Example Usages}
Learning to use a toolkit often does not work by reading the Javadoc. Therefore, we provide a simple example how to setup your first project using the toolkit, eclipse and maven. Afterwards we provide a short overview of some important modules and how you can use them. All examples are intended as entry points to facilitate the first use of the toolkit. Many components are too mighty to be explained exhaustively in small code snippets, make sure to read the Javadoc and more importantly the actual code for further information.

\subsection{Source Retriever}
The source retriever is a module that can query a number of sources such as search engines and web pages with terms and retrieve matching results.
\subsubsection{Basic Features}
Basic features are:
\begin{itemize}
\item Query the Google search engine (unlimited queries, top 64 results only).
\item Query Yahoo search engine (5000 queries per IP and day, top 1000 results).
\item Query Bing search engine (unlimited)
\item Query Hakia search engine.
\item Query Twitter.
\item Query Google Blog search.
\item Query Textrunner web page.
\end{itemize}
Some of the search APIs require API keys which must be specified in the config/apikeys.conf file. See \ref{sec:apikeys.conf} for more information.

\subsubsection{How To}
\label{sec:howto}
The following code snippet shows how to initialize the source retriever and get a list of (English) URLs from the Bing search engine for the exact search ``Jim Carrey''.
\begin{codelisting}
\begin{lstlisting}[frame=tb]
// create source retriever object
SourceRetriever s = new SourceRetriever();
		
// set maximum number of expected results 
s.setResultCount(10);
		
// set search result language to english
s.setLanguage(SourceRetrieve.LANGUAGE_ENGLISH);
		
// set the query source to the Bing search engine 
s.setSource(SourceRetrieverManager.BING);
		
// search for "Jim Carrey" in exact match mode (second parameter = true)
ArrayList<String> resultURLs = s.getURLs("Jim Carrey", true);
		
// print the results
CollectionHelper.print(resultURLs);	
\end{lstlisting}
\end{codelisting}

%\subsection{RankAggregation}

\subsection{Web Crawler}
The web crawler can be used to crawl domains or just retrieve the cleansed HTML document of a single web page.

\subsubsection{Basic Features}
Basic functionalities include:
\begin{itemize}
\item Download and save contents of a web page.
\item Automatically crawl in- and/or outbound links from web pages.
\item Use URL rules for the crawling process.
\item Extract title, description, keywords and body content of a web page.
\item Remove HTML, SCRIPT and CSS tags.
\item Find a sibling page of a given URL.
\item Switch proxies after a certain number of requests to avoid being blocked.
\end{itemize}

\subsubsection{How To}
The following code shows how to instantiate a simple crawler that starts at http://www.dmoz.org and follows all in- and outbound links. The URL of each crawled page is printed to the screen. The crawler will use 10 threads, changes the proxy after every third request and stops after having crawled 1000 pages. Instead of setting the parameters using the code, we can also specify them in the config/crawler.conf file. See \ref{sec:classification.conf} for more information.

\begin{codelisting}
\begin{lstlisting}[frame=tb]
// create the crawler object
Crawler c = new Crawler();

// create a callback that is triggered for every crawled page
CrawlerCallback crawlerCallback = new CrawlerCallback() {
	@Override
	public void crawlerCallback(Document document) {
		// TODO do something with the page
		System.out.println(document.getDocumentURI());
	}
};
c.setCrawlerCallback(crawlerCallback);

// stop after 1000 pages have been crawled (default is unlimited)
c.setStopCount(1000);

// set the maximum number of threads to 10
c.setMaxThreads(10);

// the crawler should automatically use different proxies
// after every 3rd request (default is no proxy switching)
c.setSwitchProxyRequests(3);

// set a list of proxies to choose from
List<String> proxyList = new ArrayList<String>();
proxyList.add("83.244.106.73:8080");
proxyList.add("83.244.106.73:80");
proxyList.add("67.159.31.22:8080");
c.setProxyList(proxyList);

// start the crawling process from a certain page,
// true = follow links within the start domain
// true = follow outgoing links
c.startCrawl("http://www.dmoz.org/", true, true);
\end{lstlisting}
\end{codelisting}

\subsection{FAQ Extractor}
The FAQ extractor can extract question-answer pairs from several structured frequently asked questions pages on websites. The usage is quite simple as shown in the following listing.
\begin{codelisting}
\begin{lstlisting}[frame=tb]
// create a list of question answer pairs
ArrayList<QA> qas = null;

// the URL that contains the FAQ
String url = "http://blog.pandora.com/faq/";

// start extracting question and answers from the URL
qas = QAExtractor.getInstance().extractFAQ(url);

// print the extracted questions and answers
CollectionHelper.print(qas);
\end{lstlisting}
\end{codelisting}

\subsection{Fact Extraction}
The Fact extractor can be used to detect facts in tables on web pages given a URL and optionally a small set of seed attribute names that help the extractor.
\begin{codelisting}
\begin{lstlisting}[frame=tb]
// the URL of the facts
String url = "http://en.wikipedia.org/wiki/Nokia_N95";

// the concept of the attributes
Concept c = new Concept("Mobile Phone");

// a small list of seed attributes
HashSet<Attribute> seedAttributes = new HashSet<Attribute>();
int attributeType = Attribute.VALUE_STRING;
seedAttributes.add(new Attribute("Second camera", attributeType, c));
seedAttributes.add(new Attribute("Memory card", attributeType, c));
seedAttributes.add(new Attribute("Form factor", attributeType, c));

// detect the facts using the seeds from the URL
ArrayList<Fact> detectedFacts = null;
detectedFacts = FactExtractor.extractFacts(url, seedAttributes);
		
// print the extracted facts
CollectionHelper.print(detectedFacts);
\end{lstlisting}
\end{codelisting}

\subsection{Web Page Classification}
The Web Page Classification module can be used to classify web pages by their URL or their full content.

\subsubsection{Basic Features}
The Web Page Classification module has the following basic features:
\begin{itemize}
\item Classify web pages by its URL only.
\item Classify web pages by its full content.
\item Use a combination of URL and full content for classification.
\item Learn, test and reuse models.
\item Simple one-category classification.
\item Hierarchical classification.
\item Multi-category classification (tagging).
\item All algorithms are language independent.
\end{itemize}

\subsubsection{How To}
This section describes how to prepare training and testing data to learn a model and test the classifier.

\paragraph{Preparing the Training/Testing Data}
The data can be specified in a simple text file. There are three classification options, namely, one-category classification, hierarchical classification and multi-category classification. They all require a similar structure of the data.

\subparagraph{One-Category Classification}
We write one URL and one category separated with a single space on each line. For example:
\begin{verbatim}
http://www.google.com search
http://www.fifa.com sport
http://www.oscars.com entertainment
\end{verbatim}

\subparagraph{Hierarchical Classification}
We write one URL and multiple categories separated with a single space on each line. The categories must be in the correct order, so the first category is the main one, all following are subcategories of each other. For example:
\begin{verbatim}
http://www.google.com search search_engine
http://www.fifa.com sport team_sports soccer 
http://www.oscars.com entertainment movies awards usa
\end{verbatim}

\subparagraph{Multi-Category Classification}
We write one URL and multiple categories separated with a single space on each line. The order of the categories (tags) does not matter. For example:
\begin{verbatim}
http://www.google.com search image_search video_search
http://www.fifa.com soccer sport free_time fun ball_game results to_read
http://www.oscars.com entertainment movies films awards watch video stars
\end{verbatim}

\paragraph{Building the Model}
The model is an internal representation of the learned data. After learning a model, a classifier can applied to unseen data. We now have prepared the training and testing data so we can now learn the models. The results of the test will be printed to the console and written to a log file under data/logs. The classifier is saved as a lucene index or a database under the name ``dictionary\_Xclassifier\_Y'' where X is ``url'', ``fullpage'' or ``combined'' and Y is 1 (one-category), 2 (hierarchical) or 3 (multi-category). The model will be written to data/models. How the file is saved can be configured in the config/classification.conf file. See \ref{sec:classification.conf} for more information.

\begin{codelisting}
\begin{lstlisting}[frame=tb]
// create a classifier mananger object
ClassifierManager classifierManager = new ClassifierManager();

// use 80% of the data in the training/testing file as training data
// the rest is used for testing
classifierManager.setTrainingDataPercentage(80);

// specify location of training/testing file
String ttFile = "training_testing_file.txt";

// build and test the model
// the second parameter specifies that we want to use URL features only
// the third parameter specifies that we want to use 
// multi-category classification (tagging)
// the last parameter is set to true in order to train not just test it
classifierManager.trainAndTestClassifier(ttFile,
                                         WebPageClassifier.URL, 
                                         WebPageClassifier.TAG,
                                         true);
\end{lstlisting}
\end{codelisting}

\paragraph{Using the Model}
After we trained a model for a classifier we can apply it to unseen data. Let's use the model we trained for a URL classifier with tagging:

\begin{codelisting}
\begin{lstlisting}[frame=tb]
// create the URL classifier
WebPageClassifier classifier = new URLClassifier();

// create a classification document
ClassificationDocument classifiedDocument = null;

// the web page to be classified
String url = "http://en.wikipedia.org/wiki/Computer";

// use the classifier to classify a web page to multiple categories
// using URL features only
classifiedDocument = classifier.classify(url, WebPageClassifier.TAG);

// print out classification results
System.out.println(classifiedDocument);
\end{lstlisting}
\end{codelisting}

\subsection{Helpers}
The toolkit contains many helper functionalities for reoccurring tasks in the tud.iir.helper package.
The following code snippet shows several sample usages of some of the functions.

\begin{codelisting}
\begin{lstlisting}[frame=tb]
// sort a map by its value in ascending order (2nd parameter = true)
Map m = CollectionHelper.sortByValue(map, true);

// reverse a list
List l = CollectionHelper.reverse(list);

// print the contents of a collection
CollectionHelper.print(collection);

// get the runtime of an algorithm and print it (2nd parameter = true)
long startTime = System.currentTimeMillis();
for (int i = 0; i < 10000; i++) {
	int c = i * 2;
}
DateHelper.getRuntime(t1, true);

// (de) serialization of objects
FileHelper.serialize(obj, "obj.ser");
Object obj = FileHelper.deserialize("obj.ser");

// rename, copy, move and delete files
FileHelper.rename(new File("a.txt"), "b.txt");
FileHelper.copyFile("src.txt", "dest.txt");
FileHelper.move(new File("src.txt"), "dest.txt");
FileHelper.delete("src.txt");

// get files from a folder
File[] files = FileHelper.getFiles("folder");

// zip and unzip a text
FileHelper.zip("text", "zipFile.zip");
String t = FileHelper.unzipFileToString("zipFile.zip");

// perform some action on every line of an ASCII file
final Object[] obj = new Object[1];
obj[0] = 1;

LineAction la = new LineAction(obj) {
  
    @Override
    public void performAction(String line, int lineNumber) {
        System.out.println(lineNumber + ": " + line + " " + obj[0]); 
    }
}
FileHelper.performActionOnEveryLine(filePath, la);

// round a number with a number of digits
double r = MathHelper.round(2.3333, 2);

// calculate n-grams of a string
Set<String> nGrams = StringHelper.calculateNGrams("abcde", 3);

// English singular word to plural
String p = StringHelper.wordToPlural("city");

// remove HTML tags
String r = StringHelper.removeHTMLTags("<a>abc</a>",
                                       true, true
                                       true, true);

// trim a string
String t = StringHelper.trim(" _to trim++++");

// get singular or plural of an English word
String plural = StringHelper.wordToPlural("city");
String singular = StringHelper.wordToSingular("cities");

// reverse a string
String r = StringHelper.reverse("abc");

// encode and decode base64
String e = StringHelper.encodeBase64("abc");
String d = StringHelper.decodeBase64(e);

\end{lstlisting}
\end{codelisting}

\section{Reference Libraries}
The TUDIIR Toolkit makes excessive use of third party libraries. We do not intend to re-implement code but rather to built on it and create something superior. Here an incomplete list of libraries the toolkit uses:
\begin{itemize}
\item Apache Commons \cite{apachecommons} for many standard tasks in string and number manipulation and more.
\item Fathom \cite{fathom} to measure readability of English text.
\item iText \cite{itext} for creating PDF documents.
\item Jena \cite{jena} for reading and writing ontology files.
\item jYaml \cite{jyaml} to read and write YAML files.
\item Log4j \cite{log4j} for logging.
\item Lucene \cite{lucene} for indexing and making learned models persistent.
\item NekoHTML \cite{nekohtml} to clean up the HTML of web pages in order to process them correctly.
\item SimMetrics \cite{simmetrics} to calculate similarities of strings.
\item Twitter4j \cite{twitter4j} to query the Twitter API.
\item Weka \cite{hall2009weka} for machine learning.
\end{itemize}

\section{History}
The foundation of the toolkit code came out of the WebKnox project\cite{webknox} that was started in 2008.% Now, components of the Aletheia project\footnote{\url{http://www.aletheia-projekt.de}} and the Effingo project\footnote{\url{http://www.effingo.de}}

The code is in development by students of the Dresden University of Technology. Contributors are:
\begin{itemize}
\item Christopher Friedrich
\item Martin Gregor
\item Philipp Katz
\item David Urbansky
\item Robert Willner
\item Martin Werner
\end{itemize}

\chapter{References}
\bibliographystyle{abbrv}
\bibliography{references}

\end{document}
